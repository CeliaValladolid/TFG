\capitulo{1}{Objetivos}

En este apartado  se explicarán de forma precisa y concisa cuales son los objetivos que se persiguen con la elaboración de este proyecto. 

\subsection{Objetivo general}
El objetivo general de este proyecto es simular la funcionalidad de una válvula ventriculoperitoneal controlada a través de un sensor de presión a través de un prototipo.
\subsection{Objetivos específicos}
\begin{enumerate}
    \item Investigar y documentar las diversas causas que pueden conducir al desarrollo de hidrocefalia, así como los diferentes tipos que hay, síntomas y complicaciones asociados a la patología, técnicas de diagnóstico empleadas para su detección y control y posibles opciones de tratamiento.
    \item Comprender tanto la enfermedad como al paciente que la padece para poder desarrollar un dispositivo que cumpla con todas las funcionalidades necesarias.
    \item Realizar una revisión bibliográfica sobre estudios y trabajos relacionados con este proyecto para identificar futuras mejoras del dispositivo.
    \item Aplicar el uso de herramientas empleadas a lo largo del grado como LateX para la redacción de este documento y GitHub para el seguimiento del trabajo y gestión del código.
    \item Emplear programas como BioRender para generar imágenes científicas.
    \item Buscar componentes electrónicos adecuados para llevar a cabo el prototipo, comparando unos con otros y seleccionando entre todo lo disponible en el mercado los que mejor se adecuen a nuestras necesidades.
    \item Emplear Arduino para controlar el prototipo.
    \item Medir y analizar los costes asociados a la elaboración de este proyecto.
    \item Analizar la viabilidad legal del prototipo.
    \item Diseñar prototipos de la interfaz de usuario de una futura aplicación.
\end{enumerate}








