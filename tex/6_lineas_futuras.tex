\capitulo{6}{Líneas de trabajo futuras}

Este apartado está dedicado a exponer algunas características y funcionalidades que han ido surgiendo a lo largo de la realización de este proyecto. El objetivo de todas ellas es mejorar aspectos como la seguridad, eficacia y versatilidad de este futuro dispositivo, consiguiendo así una solución más avanzada y completa.

Como he comentado en apartados anteriores, se trata de un dispositivo invasivo, colocado en el interior del cráneo a través de un procedimiento quirúrgico para monitorizar directamente la presión de LCR en mmHg\footnote{El milímetro de mercurio (mmHg) es una unidad de presión manométrica, es decir, medida en relación a la presión del ambiente con respecto a la atmósfera \cite{presion}.} Uno de los objetivos principales en cuanto al diseño es la miniaturización del dispositivo, conseguir que sea lo más pequeño posible para la máxima comodidad del paciente.

%https://www.essalud.gob.pe/ietsi/PETITORIO_DE_MATERIAL_MEDICO/pdf/MM-538.pdf 
Respecto a los materiales empleados, deben de ser compatibles con la realización de una RMN, es decir, no se pueden emplear metales ya que los imanes de la resonancia se sentirán atraídos por estos, interfiriendo posteriormente en los resultados. Solo algunos como el cobre, cobalto, cromo y titanio son metales seguros que no distorsionan el resultado. El titanio es un metal que se utiliza de forma habitual en dispositivos médicos implantables debido a que presenta una elevada resistencia a la corrosión de los fluidos corporales. Es biocompatible con los tejidos del organismo, no los daña ni produce toxicidad \cite{titanio}. Para los catéteres, el material más empleado es la silicona. Este material al igual que los demás ha sido previamente testado y es de gran calidad, hipoalergénico\footnote{Un material hipoalergénico o hipoalérgico es aquel que produce una reacción alérgica escasa o nula \cite{hipoalergenico}.}, antibacteriano, flexible, resistente y libre de látex y de toxinas \cite{silicona}. Una caracterítica destacable tanto de los catéteres como de la válvula es que sean radiopacos, es decir, que ofrezcan resistencia al ser atravesados por rayos X, u otra forma de radiación ionizante, de tal forma que en el resultado se vean de color blanco. Esta cualidad resulta interesante ya que permite a los neurocirujanos visualizar estos dispositivos de forma clara durante intervenciones guiadas radiológicamente \cite{radiopaco}.

Este sistema de derivación ventriculoperitoneal también llevaría incorporado un reservorio y un dispositivo de control antisifón. En la Figura \ref{fig:valvula} podemos observar lo que sería un reservorio, la esfera de donde parte la flecha. Su función es almacenar LCR por si se requiere una muestra de este para examinarlo, sin necesidad de realizar una compleja cirugía para conseguirlo, basta con realizar una punción.

El dispositivo de control antisifón es de vital importancia para solventar uno de los principales problemas que presentan los sistemas de derivación: el hiperdrenaje de LCR por la acción de la fuerza de gravedad, que ocurre cuando el paciente cambia de posición horizontal a vertical \cite{derivacion}. De esta forma intentaremos drenar en todo momento la cantidad adecuada, evitando así un drenaje excesivo o deficiente. 

Por último, este dispositivo contará con una aplicación móvil. Los datos recogidos en tiempo real por el sensor, se transmitirán a un servidor en la nube, el cual será accesible por la aplicación\footnote{Esta idea se basa en el proyecto de Curitiba \cite{curitibaa}, explicado en el apartado de Introducción, Estado del arte y trabajos y relacionados.}. A través de esta, se podrán modificar parámetros tales como la apertura de la válvula y los umbrales de alarma. El establecimiento de estos umbrales está pensado para conseguir una anticipación a posibles complicaciones, mediante el envío de una notificación de alerta al paciente y al profesional sanitario cuando estos umbrales sean superados, tanto para presiones demasiado altas como demasiado bajas. El médico será el encargado de establecer estos valores en función de los requerimientos de cada paciente consiguiendo así un ajuste más personalizado. Todos los datos recopilados a lo largo del tiempo estarán almacenados en historiales con el objetivo de poder estudiarlos, identificando patrones y cambios de presión. Con esta información se podrán realizar ajustes más informados y personalizados mejorando la gestión del paciente a largo plazo e intentado prevenir contratiempos futuros.








