\apendice{Anexo de sostenibilización curricular}

\section{Reflexión personal sobre los aspectos de sostenibilidad}

El desarrollo de tecnologías médicas debe tener en cuenta la sostenibilidad, y este Trabajo de Fin de Grado centrado en el desarrollo de una válvula de derivación ventriculoperitoneal que integra un sensor para medir los valores de presión intracraneal y controlar la apertura de la válvula en función de las lecturas que realice, aborda numerosos aspectos relacionados con la sostenibilidad y los derechos fundamentales. A continuación, se examinarán estos elementos y las habilidades adquiridas durante la elaboración del presente proyecto, en relación con las disposiciones del \textbf{Real Decreto 822/2021} \cite{anexoH-realdecre}.

\subsection{Derechos humanos y derechos fundamentales}
El proyecto demuestra un respeto profundo por los \textbf{derechos humanos}\footnote{Los derechos humanos, recogidos en la Declaración Universal de los Derechos Humanos (DUDH), son los derechos que tiene una persona desde su nacimiento por el simple hecho de existir y se reconocen a todas las personas del mundo. Los derechos fundamentales tienen un alcance nacional, y dependiendo del país, pueden cambiar.} \cite{dere-hum} y los \textbf{derechos fundamentales} \cite{dere-funda}, que son fundamentales para el avance tecnológico, especialmente en el ámbito médico. El objetivo del prototipo desarrollado en este trabajo es mejorar significativamente la calidad de vida de los pacientes con hidrocefalia, ofreciendo un método efectivo, preciso y automatizado para controlar la presión intracraneal del paciente. Este método fomenta los valores democráticos y la equidad al garantizar que todos los pacientes, independientemente de su origen o condición, puedan beneficiarse de las innovaciones médicas, respetando su dignidad y sus derechos.
    
\subsection{Igualdad de género y no discriminación}
En el desarrollo de este proyecto, se ha adoptado una perspectiva inclusiva que respeta la igualdad de género y la no discriminación. La \textbf{Ley Orgánica 3/2007, de 22 de marzo, para la igualdad efectiva de mujeres y hombres} \cite{ley-igualdad} subraya la importancia de eliminar cualquier forma de discriminación. En este sentido, el proyecto ha asegurado que tanto el desarrollo como el posterior uso de la válvula sean accesibles para todos los pacientes, sin importar su género, edad, origen étnico, o cualquier otra condición. Además, si en un futuro se retoma el proyecto incluyendo las mejoras citadas en apartados anteriores, la composición del equipo de desarrollo y las oportunidades de participación serán equitativas, fomentando y promoviendo un entorno de trabajo inclusivo y diverso.

\subsection{Accesibilidad universal y diseño para todas las personas}
El respeto a los principios de accesibilidad universal y diseño para todas las personas es otro componente a tener en cuenta en este proyecto. Esta válvula ha sido diseñada teniendo en cuenta las necesidades de personas que padecen hidrocefalia, incluidos aquellos con discapacidades. La futura aplicación para poder monitorizar al paciente contará con una interfaz sencilla, siendo fácil de usar y estando adaptada a los requerimientos de los pacientes. Este enfoque asegura que el dispositivo junto con la aplicación no solo sean efectivos en su función principal sino que también sean accesibles para cualquier persona que lo necesite, promoviendo una mayor inclusión en la atención médica.

\subsection{Sostenibilidad y cambio climático}
El tratamiento de la sostenibilidad y del cambio climático es una potente consideración hoy en día, de acuerdo con la \textbf{Ley 7/2021, de 20 de mayo, de cambio climático y transición energética} \cite{ley-cc}. La selección de materiales y la metodología de desarrollo deben ir orientados a minimizar el impacto ambiental, empleando componentes eficientes y duraderos para evitar ser reemplazados a corto plazo. Estos serían algunos aspectos a tener en cuenta:
\begin{enumerate}
    \item \textbf{Selección de materiales sostenibles}: optar por materiales reciclables, biodegradables o que necesiten menos energía para su fabricación.
    \item \textbf{Eficiencia energética}: enfocar el diseño de los dispositivos hacia la reducción del consumo de energía, empleando fuentes de energía renovable.
    \item \textbf{Durabilidad}: crear dispositivos duraderos, con componentes resistentes que no necesiten ser cambiados de forma frecuente.
    \item \textbf{Reducción de residuos}: fomentar el reciclaje y la reutilización de componentes, diseñando productos modulares para que en el caso de tener que repararlos no se generen tantos residuos ya que se podrá cambiar únicamente el componente dañado.
    \item \textbf{Compromiso e investigación}: mantener un compromiso con la sostenibilidad y la disminución del impacto climático invirtiendo en investigación para desarrollar nuevas tecnologías a través de prácticas más respetuosas con el medio ambiente.
\end{enumerate}

A lo largo del desarrollo de este proyecto se han adquirido y aplicado diferentes competencias acerca de la sostenibilidad. Entre ellas destacan la capacidad para desarrollar un producto accesible para todo el mundo independientemente del género, edad, origen étnico o presencia de alguna discapacidad promoviendo la igualdad y la no discriminación, y la adquisición de conocimiento referente a las leyes vigentes que tratan del cambio climático y promueven la sosteniblidad. Este proyecto sirve para sentar las bases que se deben seguir en un futuro para desarrollar dispositivos médicos tecnológicos que sean duraderos, eficientes y respetuosos con el medio ambiente.